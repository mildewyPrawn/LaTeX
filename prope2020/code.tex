\documentclass[spanish,12pt,letterpaper]{article}

\usepackage[spanish]{babel}
\usepackage[utf8]{inputenc}
\usepackage{authblk}

\usepackage{minted} % Código bonito

\title{Propedéutico 2020 \\ \LaTeX}
\author{Victor Zamora Gutiérrez \\ Karla Socorro García Alcántara\\
  Galeana Araujo Emiliano}
\affil{Facultad de ciencias, UNAM}
\date{\today}

\begin{document}

\maketitle

minted es un paquete para poner código y que se vea bonito, con colores...\\

\section{Instalación}
\begin{itemize}
\item arch linux $\rightarrow$ pacman -S minted
\item ubuntu $\rightarrow$
  \begin{itemize}
  \item sudo apt-get install texlive-latex-extra
  \item sudo apt-get install python-pygments
  \end{itemize}
\end{itemize}

\section{Compilar}
Ahora compilamos de la siguiente manera:\\
%% \verb!pdflatex -synctex=1 -interaction=nonstopmode --shell-escape %.tex!
\verb!pdflatex -shell-escape archivo.tex!

\section{Ejemplos}
\subsection{Python}
\begin{minted}[breaklines]{Python}
  # Filtro de mica morada
  def filtroMicaMorada(imagen, nueva):
    rgb = imagen.convert('RGB')
    pixels = nueva.load()
    for i in range(imagen.size[0]):
        for j in range(imagen.size[1]):
            r,g,b = rgb.getpixel((i,j))
            pixels[i,j] = (b,0,r)
    return nueva
\end{minted}

\subsection{C}
\begin{minted}[breaklines]{C}
/**
 * direccion
 * Imprime la direccion actual
 */
void direccion() {
  char cwd[1024];
  //getcwd determina el nombre de la ruta del directorio de trabajo
  // y lo almacena en el búfer.
  getcwd(cwd, sizeof(cwd));
  //getenv busca la cadena de entorno a la que apunta el nombre y devuelve el valor asociado a la cadena.
  char* username = getenv("USER");
  //Se imprime en este caso kevin@maquina ~ \home\kevind\Documentos\Tarea3
  printf("%s@maquina ~ %s", username, cwd);
}
\end{minted}

\subsection{Java}
\begin{minted}[breaklines]{Java}
  // Método que imprime algo con un arreglo
  public static void imprime(int[] sec){
    int i = 0;
    String s = "[" + sec[i] + ", ";
      while(++i < sec.length){
        s += sec[i] + ", ";
      }
      System.out.println(s + sec[--i]  +"]");
  }
\end{minted}


\subsection{Haskell}

\begin{minted}[breaklines]{Haskell}
-- | sumaGauss. Funcion que representa la formula de Gauss.
sumaGauss :: Int -> Int
sumaGauss x = (x*(x+1)) `div` 2
\end{minted}

\end{document}
