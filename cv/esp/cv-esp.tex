\documentclass[10pt,A4]{article}	

\usepackage[T1]{fontenc}
\usepackage[a4paper]{geometry}
\usepackage[hidelinks]{hyperref}
\usepackage[utf8]{inputenc}
\usepackage{calc}
\usepackage{fancyhdr}
\usepackage{fontawesome}
\usepackage{moresize}
\usepackage{tabularx}
\usepackage{tikz}
\usepackage{xifthen}
\usepackage{titling}
\usepackage{titlesec}
\usepackage{multicol}
\usepackage{setspace}

%% for example, change the margins to 2 inches all round
\geometry{top=2cm, bottom=2cm, left=0.5cm, right=0.5cm} 	
%% use customized header
\pagestyle{fancy}
%% less space between header and content
\setlength{\headheight}{0pt}
%% customize header entries
\lhead{}
\rhead{}
\chead{}
%indentation is zero
\setlength{\parindent}{0mm}
% for drawing graphics and charts
\usetikzlibrary{shapes, backgrounds}

\newcommand*{\vcenteredhbox}[1]{\begingroup
  \setbox0=\hbox{#1}\parbox{\wd0}{\box0}\endgroup}

\newcommand{\icon}[2]{\colorbox{black}{\makebox(#2, #2){\textcolor{white}{\large\csname fa#1\endcsname}}}}	%icon shortcut
\newcommand{\icontext}[3]{ 						%icon with text shortcut
  \vcenteredhbox{\icon{#1}{#2}}\hspace{0.2cm}\vcenteredhbox{\textcolor{black}{#3}}
}

\titlespacing{\section}{1cm}{1cm}{.12cm}
\titlespacing{\subsection}{1cm}{1cm}{.01cm}


%----------------------------------------------------------------------------------------
% 	HEADER
%----------------------------------------------------------------------------------------
% remove top header line
\renewcommand{\headrulewidth}{0pt}
%remove botttom header line
\renewcommand{\footrulewidth}{0pt}
%remove pagenum
\renewcommand{\thepage}{}
%remove section num
\renewcommand{\thesection}{}
%% family font
\renewcommand*\familydefault{\sfdefault}

%----------------------------------------------------------------------------------------
%	custom sections
%----------------------------------------------------------------------------------------
\titleformat{\section}
{\large\bfseries\uppercase}
{}
{0em}
{}
{}

\titleformat{\subsection}
{\bfseries\Large}
{\hspace{-.75cm}}
{0em}
{}
{}

%----------------------------------------------------------------------------------------
% MY COMMANDS
%----------------------------------------------------------------------------------------
\newcommand{\entry}[4]{  
  \begin{minipage}[t]{.15\textwidth}
    \hfill \textsc{#1}\\
  \end{minipage}
  \hfill\vline\hfill
  \begin{minipage}[t]{.80\textwidth}
    #2\\
    \onehalfspacing
    \textit{#3}\\
    \footnotesize{#4}
  \end{minipage}\\
}

\newcommand{\UNAM}{\textit{UNAM}}
\newcommand{\FC}{\textit{Facultad de Ciencias}}

%%%%%%%%%%%%%%%%%%%%%%%%%%%%%%%%%%%%%%%%%%%%%%%%%%%%%%%%%%%%%%%%%%%%%%%%%%%%%%%%
%%%%%%%%%%%%%%%%%%%%%%%%%%%%%%%%%%%%%%%%%%%%%%%%%%%%%%%%%%%%%%%%%%%%%%%%%%%%%%%%
%%%%%%%%%%%%%%%%%%%%%%%%%%%%%%%%%%%%%%%%%%%%%%%%%%%%%%%%%%%%%%%%%%%%%%%%%%%%%%%%

\begin{document}
  \begin{center}
    \colorbox{black}{{\HUGE\textcolor{white}{\textbf{\MakeUppercase{Emiliano Galeana Araujo}}}}}
  
    \vspace{1mm}\Large{Ciencias de la computación y Matemáticas}
  \end{center}

\begin{minipage}[t]{0.50\textwidth}
  \icontext{MapMarker}{10}{CDMX, Mexico}

  \icontext{Phone}{10}{+52 5518474280}

  \icontext{At}{8}{\href{mailto:galeanaara@ciencias.unam.mx}{galeanaara@ciencias.unam.mx}}
\end{minipage}
\begin{minipage}[t]{0.50\textwidth}
  \icontext{Gitlab}{10}{\href{https://gitlab.com/mildewyPrawn}{gitlab.com/mildewyPrawn}}

  \icontext{Github}{10}{\href{https://github.com/mildewyPrawn}{github.com/mildewyPrawn}}
\end{minipage}%

% manage space by reducing font size
%----------------------------------------------------------------------------------------
%	SKILLS AND TECHNOLOGIES
%----------------------------------------------------------------------------------------
\begin{minipage}[c]{0.55\linewidth}

  \section{Experiencia laboral}
\entry
    {2020 - presente}
    {\UNAM, \FC}
    {Ayudante de laboratorio para cursos de la carrera de ciencias de la computación:}
    {Propedéutico, Arquitectura de Computadoras, Estructuras de Datos, Estructuras Discretas, Lógica Computacional}

\entry
    {2020-2021}
    {Conacyt, Centro Geo}
    {Auxiliar de laboratorio en Laboratorio Nacional de
      Geointeligencia. En equipo diseñamos un curso (Con ejercicios)
      de python para personas de nuevo ingreso al laboratorio.}
    {python, jupyter}
    
\entry
    {Fall 2018}
    {UNAM-PAPIME102117, \FC}
    {Programa de becas. Diseñé y escribí un solucionario para el curso de lenguajes de
      programación a cargo del Dr. Favio E. Miranda Perea}
    {\LaTeX}

\section{Proyectos}
\entry
    {2020}
    {Parser de Música}
    {Implementé en Haskell una especie de parser para música, toma
      como input un archivo con notas musicales y produce sonido.}
    {haskell, ffplay, git}
      
\entry
    {2020}
    {Heurística de Optimización para resolver TSP}
    {Implementé el recorrido simulado para resolver el problema del
      agente viajero (TSP). Tomando datos de las ciudades más
      populares.}
    {golang, sql, git}
    
\entry
    {2019}
    {Desarrollé una aplicación web para el registro de
      eventos y asistentes.}
    {Proyecto escolar. Lideré el desarrollo del
      proyecto para la materia de Ingeniería de Software. Se realizó
      (Además de la implementación), documentación, vistas, diagramas,
      pruebas.}
    {python, django, postgresql, git}
    
\entry
    {2019}      
    {Cierre Convexo/Barrido de línea}
    {Implementé algoritmos para la materia de geometría
      computacional.}
    {java, c++, git}
               
\entry
    {2019}
    {Juego de Othello}
    {Implementación de una inteligencia artificial que puede jugar
      othello (Y ganar en la mayoría de los casos).}
    {python, git, processing}

\entry
    {2020}
    {cafeCiencias}
    {Proyecto para la materia de riesgo tecnológico. Aplicación para
      el registro y apartado de alimentos en la escuela. Contribuí a:
      la implementación, documentación, vistas, pruebas, base de
      datos.}
    {python, django}

\end{minipage}
\begin{minipage}[c]{0.40\linewidth}

\section{Educación}  
\entry
    {Desde 2017}
    {Licenciatura en Ciencias de la computación.}
    {\UNAM, \FC.}
    {Promedio: 9.02}

\entry
    {Desde 2019}
    {Licenciatura en Matemáticas.}
    {\UNAM, \FC.}
    {Promedio: 8.83}

\section{Otros}
\entry
    {Desde 2019}
    {Participante de la ACM ICPC.}
    {Recibí mención honorífica.}
    {C++}
        
\entry
    {2019}
    {Mejor idea para el desafío de IBM.}
    {Ganador del desafío de IBM en el hackathon UNAMxHacks. Sistema
      para clasificar curriculums.}
    {python, cherrypy}

\entry
    {2018}
    {Curso de Introducción al diseño de bases de datos relacionales.}
    {DGTIC, \UNAM.}
    {Microsoft Access}

\entry
    {2017}
    {Curso de Introcucción a Sistemas Operativos tipo UNIX.}
    {DGTIC, \UNAM.}
    {linux (kali)}

\section{Desarrollo}
\begin{center}
  \begin{tabular}{l c}
    Java, python, haskell, git  & 4 años\\
    golang, C/C++ & 2 años\\
    lisp & 1 año\\
  \end{tabular}
\end{center}

\section{Intereses}
  Algoritmos, Geometría Computacional, Estructuras de Datos,
  Inteligencia Artificial, Programación Declarativa.

\section{Idiomas}
  \begin{center}
  \begin{tabular}{l l}
  Español & Nativo\\
  Inglés & B1\\
  Francés & Básico\\
  \end{tabular}
\end{center}

\end{minipage}
\end{document}
